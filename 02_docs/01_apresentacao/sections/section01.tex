\section{Construção dos modelos} 

% ----------------------------

\begin{frame}{\textit{PE Model}: Uma breve apresentação das equações primitivas}
		
	\begin{center}
		\large Apresentar brevemente as equações e motivações etc. Confirir o que está no disc
	\end{center}
\end{frame}

% ----------------------------

\begin{frame}{\textit{PE Model}: Características do fluido}
	\begin{itemize}
		\item \textbf{Homogêneo.} A densidade do fluido é uniforme em todo volume;
		\item \textbf{Incompressível.} O volume não muda quando submetido à pressão ($\nabla \cdot V = 0$)
	\end{itemize}
\end{frame}

% ----------------------------

\begin{frame}{\textit{PE Model}: O modelo de água rasa}
		
	Fazendo as devidas alterações notacionais em \cite{salmon1998}, temos:
	\begin{align}
		\frac{\partial V}{\partial t} + (V \cdot \nabla)V + f \mathbf{k} \times V & = -g \nabla z \label{eq:agua-rasa-1} \\
		\frac{\partial z}{\partial t} + \nabla \cdot (z V)                        & = 0 \label{eq:agua-rasa-2}           
	\end{align}
		
	\begin{small}
		Onde:
		\begin{multicols}{2}
			\begin{itemize}
				\item $t$: tempo;
				\item $\mathbf{r}$: vetor de posição inicial;
				\item $V(t,r)$: Velocidade horizontal;
				\item $z(t,r)$: altura da superfície;
				\item $f$: parâmetro de Coriolis;
				\item $g$: aceleração da gravidade;
				\item $\mathbf{k}$: vetor da vertical.
			\end{itemize}
		\end{multicols}
	\end{small}
\end{frame}
% ----------------------------

\begin{frame}{\textit{PE Model}: O modelo de água rasa modificado}
	\begin{small}
		No artigo, nos é apresentado as seguintes equações:
		\begin{align}
			\frac{\partial V}{\partial t} & = - ( V \cdot \nabla)V - f \mathbf{k} \times V - g \nabla z + \nu \nabla^2\mathbf{V} \label{eq:agua-rasa-modificada-1}     \\
			\frac{\partial z}{\partial t} & = - (V \cdot \nabla)(z - h) - (H + z - H)\nabla \cdot \mathbf{V} + \kappa \nabla^2 z + F \label{eq:agua-rasa-modificada-2} 
		\end{align}
	\end{small}
	\begin{scriptsize}
		Onde:
		\begin{multicols}{2}
			\begin{itemize}
				\item $t$: tempo
				\item $\mathbf{r}$: vetor de posição inicial;
				\item $H$: profundidade média do fluido;
				\item $h(r)$: variação da superfície topológica;
				\item $V(t,r)$: Velocidade horizontal;
				\item $z(t,r)$: altura da superfície;
				\item $f$: parâmetro de Coriolis;
				\item $g$: aceleração da gravidade;
				\item $F$: forças externas;
				\item $\kappa$: coeficiente de difusão viscosa;
				\item $\nu$: coeficiente de difusão térmica;
				\item $\mathbf{k}$: vetor da vertical.
			\end{itemize}
		\end{multicols}
	\end{scriptsize}
\end{frame}

% ----------------------------

\begin{frame}{\textit{PE Model}: Sobre os processos de difusão}
	Nas equações \eqref{eq:agua-rasa-modificada-1} e \eqref{eq:agua-rasa-modificada-2}, temos dois processos de difusão:
	\begin{enumerate}
		\item \textbf{Difusão viscosa:} transferência do momento entre partes do fluido devido à viscosidade \textit{(exemplo: mel)};
		\item \textbf{Difusão térmica:} Transferência de calor por condução entre regiões com diferentes temperaturas.
	\end{enumerate}
		
	\begin{small}
	    \textit{Observação: Ambos os processos tendem a uniformizar suas respectivas propriedades.}
	\end{small}
\end{frame}

% ----------------------------
\begin{frame}{\textit{PE Model}: Algumas observações das equações \eqref{eq:agua-rasa-modificada-1} e \eqref{eq:agua-rasa-modificada-2}}
	\begin{enumerate}
		\item A média horizontal de $h$ e $z$ é zero;
		\item $V(t,r)$ e $z(t,r)$ são amortecidas pelo processo de difusão de pequenas escala, este fato auxilia na simulação de fenômenos atmosféricos;
		\item O ``efeito $\beta$'', efeito que indica como o movimento do fluido é afetado pelas alterações espaciais do parâmetro Coriolis, é suprimido da equação \eqref{eq:agua-rasa-modificada-2}, através da escolha da topografia. Tal decisão baseia-se no artigo \cite{von_arx1952} que prova o fato teoricamente e laboratorialmente. 
	\end{enumerate}
\end{frame}

% ----------------------------

\begin{frame}{\textit{PE Model}: Modelo de água rasa}
    \begin{center}
        Colocar o desenho aqui
    \end{center}
\end{frame}


% ----------------------------
\begin{frame}{\textit{PE Model}: A formação de novas equações}
	A partir da \textit{Decomposição de Helmholtz}, decomposição que divide a \textit{parte rotacional} e a parte \textit{divergente}, aplicada a equação \eqref{eq:agua-rasa-modificada-2}, temos:
	
	\begin{equation}
		V = \nabla\chi + \mathbf{k} \times \nabla \psi \label{eq:decomposicao-helmholtz}
	\end{equation}
	
	Onde:
	\begin{itemize}
		\item $\chi$: potencial de velocidade (\textit{parte divergente})
		\item $\psi$: função corrente (\textit{parte rotacional})
		\item $\mathbf{k}$: vetor unitário vertical
	\end{itemize}
\end{frame}

% ----------------------------

\begin{frame}{\textit{PE Model}: A formação de novas equações}
   A partir da equação \eqref{eq:decomposicao-helmholtz} e \eqref{eq:agua-rasa-modificada-1}, obtemos as duas seguintes equações:
   \begin{small}
       \begin{align}
           \frac{\partial \nabla^2 \chi}{\partial t} &= -\frac{1}{2}\nabla^2(\nabla \chi \cdot \nabla \chi) - \nabla \chi \cdot \nabla(\nabla^2\psi) \times \mathbf{k} + \nabla^2(\nabla \chi \cdot \nabla \psi \times \mathbf{k}) \nonumber \\
           &\quad + \nabla \cdot (\nabla^2\psi\nabla\psi) - \frac{1}{2}\nabla^2(\nabla \psi \cdot \nabla \psi) + \nu\nabla^4\chi + f\nabla^2\psi - g\nabla^2z \label{eq:equacao-basica-1} \\
           \frac{\partial \nabla^2 \psi}{\partial t} &= -\nabla \cdot (\nabla^2\psi\nabla \chi) - \nabla \psi \cdot \nabla(\nabla^2\psi) \times \mathbf{k} - f\nabla^2\chi + \nu\nabla^4\psi\label{eq:equacao-basica-2}
       \end{align}
   \end{small}

   Onde:
   \begin{itemize}
       \item $\frac{\partial \nabla^2 \chi}{\partial t}$: expressa a vorticidade;
       \item $\frac{\partial \nabla^2 \psi}{\partial t}$: expressa o divergente
   \end{itemize}
\end{frame}

\begin{frame}{\textit{PE Model}: A formação de novas equações}
   Realizando o mesmo processo, a partir das equações \eqref{eq:agua-rasa-modificada-2} e \eqref{eq:decomposicao-helmholtz}, obtemos a seguinte equação:
    \begin{equation}
        \frac{\partial z}{\partial t} = -\nabla \cdot (z - h)\nabla \chi - \nabla \psi \cdot \nabla(z - h) \times \mathbf{k} - H\nabla^2\chi + \kappa\nabla^2z + F \label{eq:equacao-basica-3}
    \end{equation}

    As equações \eqref{eq:equacao-basica-1}, \eqref{eq:equacao-basica-2} e \eqref{eq:equacao-basica-3} serão as equações básicas para a construção do modelo de baixa ordem
\end{frame}
