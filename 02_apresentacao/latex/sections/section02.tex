\section{Comparação entre os modelos} 

%---------------------------------------------------

\begin{frame}{Observações iniciais}

É importante destacar que: para apresentação, optei por apresentar a comparação entre os modelos tomando \textbf{apenas} a base para o critério de comparação entre os modelos.

Nas seções finais do artigo \cite{lorenz1980}, há uma análise detalhada sobre a estrutura do atrator e sua relação com a variedade invariante, incluindo suas propriedades qualitativas.
\end{frame}

%---------------------------------------------------

\begin{frame}{Modelo dissipativo forçado}
    Primeiro, vamos expor um Modelo dissipativo forçado genérico:
    \begin{equation}
        \frac{d w_i}{dt} = \sum_{j,k}^N a_{ijk} w_j w_k - \sum_j^N b_{ij} w_j + c_i \label{eq:sistema-generico}
    \end{equation}

    \begin{equation*}
        A = \sum_{i,j,k}^N a_{ijk} w_i w_j w_k \quad \land \quad B = \sum_{i,j}^N b_{ij} w_i w_j > 0
    \end{equation*}

    \begin{equation*}
         C = \sum_i^N c_i w_i \quad \land \quad R^2 = \sum_i^N w_i^2
    \end{equation*}

\end{frame}

%---------------------------------------------------

\begin{frame}{Modelo dissipativo forçado}

\begin{itemize}
    \item $A$ é um polinômio cúbico e representa as interações não lineares entre as componentes do sistema;
    \item $B$ é um polinônio quadrático e representa a dissipação do sistema;
    \item $C$ é um polinômio linear e representa o forçamento externo;
    \item $R$ é a norma euclidiana ao quadrado e representa a energia total.
\end{itemize}
E vamos definir o $A_1$ e $C_1$ como o máximo de $A$ e $C$ e $B_1$ como o mínimo de $B$
\end{frame}

%---------------------------------------------------

\begin{frame}{Condições especiais}

Os modelos de interesse do artigo \cite{lorenz1980} as seguintes condições:
\begin{enumerate}
    \item \begin{equation*}
        B_1^2 - 4A_1C_1 > 0 \label{eq:condicao-sis}
    \end{equation*}
    \item $a_{ijk} = 0 \text{, se } j=1 \quad \vee \quad k = i$
    \item \textbf{Volume zero}
    \begin{equation}
        \frac{dV}{dt} = - V \sum_i^N b_{ii} \label{eq:volume-zero}
    \end{equation}
\end{enumerate}

\end{frame}


%---------------------------------------------------

\begin{frame}{Volume zero}

A equação \eqref{eq:volume-zero} mostra que:
\begin{itemize}
    \item Como $B$ é definido positivo, temos $V \to 0$ exponencialmente.
    \item Isso implica que a dinâmica do sistema restringe-se progressivamente a conjuntos de menor volume.
    \item Qualquer superfície inicial $S$ gera uma sequência \( S_1, S_2, \dots \), cada uma envolvendo um volume menor que a anterior.
    \item No limite, o volume da sucessão de superfícies tende a zero.
\end{itemize}

\end{frame}

%---------------------------------------------------

\begin{frame}{Relacionando o modelo genérico com os modelos construídos}

\begin{enumerate}
    \item \textbf{Modelo QG.} No modelo QG, a energia total é conservada pelos termos quadráticos. Além disso, a dissipação introduzida pelos coeficientes de atrito e viscosidade age de forma análoga ao termo dissipativo do sistema genérico \eqref{eq:sistema-generico}. Assim, o modelo QG possui um atrator de \textbf{volume zero}.
    
    \item \textbf{Modelo PE.} Apesar do modelo PE não conservar a energia total, se \( F_1^2 +F_2^2+F_3^2 \) for suficientemente pequeno, a condição \eqref{eq:condicao-sis} é satisfeita. Nesse caso, as trajetórias do sistema permanecem limitadas e o atrator tem \textbf{volume zero}.
\end{enumerate}

\end{frame}

